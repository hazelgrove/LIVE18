% !TEX root = hazel-LIVE2018.tex

\section{Live Palettes}
\label{sec:palettes}


\begin{figure}[t]
\includegraphics[width=0.7\textwidth]{images/cutoffs-new.png}
\caption{An example where the teacher assigns cutoffs 
for letter grades using a domain-specific live palette.}
\label{fig:cutoffs-example}
\end{figure}

Another editor service that we are designing for integration into \Hazel 
is the \emph{live palette} service. 
Palettes, which were introduced in work on
\emph{active code completion} by \citet{ActiveCodeCompletion},  allows programmers to fill typed 
holes using specialized graphical user interfaces rather than exclusively symbolic
code. For example, a palette might allow the programmer to 
fill in a hole of matrix type using a tabular data entry 
user interface, rather than by construcing a matrix 
using the constructors of the matrix data type. \citet{ActiveCodeCompletion} elicited a large number of other
examples of data types where an alternative, graphical 
method of constructing a value of a particular type might be
useful.
Palettes in the Graphite system by \citet{ActiveCodeCompletion} 
were ephemeral, operating as a form of code completion, but projectional editors, e.g. Barista \cite{ko_barista:_2006}, often support a small number of  
palettes (there called \emph{projections}) that persist, i.e. that appear within the code itself.
They are also compositional, i.e. they also allow code to appear within the palette (e.g. as entries in the matrix user interface). 

Like Graphite, \Hazel's palette system is extensible. 
Like projectional palettes, \Hazel's palettes are also persistent and compositional. 
Uniqely, \Hazel's palette system is live: the program 
can be executed before the palette has generated code
and the hole closures associated with the hole that the 
palette is tasked with filling can be made available to 
the palette.

Let us consider an illustrative example that demonstrates
all of these qualities. 

Fig.~\ref{fig:cutoffs-example} shows a palette that allows the teacher to determine grade cutoffs by dragging markers visually along a number line, with the student's grades superimposed.

%
Each palette is required to implement a
Model-Update-View interface, following roughly the Elm architecture \cite{ElmArchitecture}. 
In the function \li{view} that generates the user
interface, the palette may request that a nested hole be
rendered using the \li{HtmlHole} constructor, choosing a
name for it (\li{id}) and specifying its type (\li{ty}). When the cursor is in this hole, the editor (Hazel) provides all of the usual editor services based on the specified type. Here, the \li{data} hole 
is filled by an expression \li{weighted_averages} of type \li{list(float)} (from the example in \autoref{fig:intro-example}, extended with some additional student data). This would have been suggested by Track 3.
%% for it (\li{hole\_spec.id}) and specifying its type
%% (\li{hole\_spec.type}).
%

At any point, the editor may invoke the palette's
\li{to_exp} function to produce the expansion, \ie{}~an expression
represented as a value of type \li{expr_ast}. The palette can refer to
the nested hole abstractly (via \li{id}) 
%% the nested hole abstractly (via its \li{hole\_spec.id})
with the \li{EHoleRef} constructor, allowing its filled
expression to be used abstractly (at type \li{ty}). The \li{to_exp} function generates a value of type \li{grade_cutoffs},
shown in \autoref{fig:intro-example-part-2}, based on the positions of the grade markers,
which can be dragged by the user. The position of these markers is tracked in the model. The palette is implemented such that manipulating any one of the cutoffs also causes the others to change if the sorting invariant would have been violated.

The palette can ask the environment to evaluate the expression
 inside a hole by referring to it abstractly and calling \li{Env.run} as shown in the body 
 of the \li{view} function. Here, the palette uses this to show orange dots for all of 
 the computed weighted averages. The palette also runs a bucketing calculation (not shown)
based on the data and the current cutoffs, and then checks
the data-driven ``fairness'' criterion that no two students
whose grades are within some small factor get assigned
different letter grades. This allows the teacher to easily choose appropriate points, and demonstrates the utility of a palette over a plain expression. 
% cutoffs and is usable for different data sets in different
% courses and in different years.
 If the hole was not filled or filled by an indeterminate value, then 
 the dots and warning would not be displayed.
