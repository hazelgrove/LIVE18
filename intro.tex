% !TEX root = hazel-LIVE2018.tex

\section{Introduction}
\label{sec:intro}

Programmers typically shift back and forth between program editing and program evaluation many times before converging upon a program that behaves as is intended. 
Live programming environments aim to support this workflow by interleaving editing and evaluation so as to   
narrow what \citet{burckhardt2013s} call the ``temporal and perceptive gap'' between these activities.
% In other words, the goal \IS to provide continuous feedback about the dynamic behavior of the program, in whole or in part,
% directly alongside the program text itself.

% For example, read-evaluate-print loops (REPLs) and derivatives thereof, like the IPython/Jupyter lab notebooks popular in data science~\cite{PER-GRA:2007}, allow the programmer to edit and immediately execute program fragments organized into a sequence of cells. 
% Spreadsheets are live functional dataflow environments, with cells organized into a grid \cite{DBLP:journals/jfp/Wakeling07}. 
% More specialized examples include live direct manipulation programming environments like SuperGlue
% \cite{McDirmid:2007}, \sns{}~\cite{sns-pldi,sns-uist}, and the tools
% demonstrated by \citet{victor2012inventing} in his lectures;
% %
% live user interface frameworks \cite{burckhardt2013s};
% %
% live image processing languages~\cite{DBLP:journals/vlc/Tanimoto90};
% %
% and live visual and auditory dataflow languages \cite{DBLP:conf/vl/BurnettAW98}, which can support live coding as a performance art.
% Editor-integrated debuggers \cite{mccauley2008debugging} and other systems that support editing run-time state, like Smalltalk environments \cite{Goldberg:1983cn}, are also live programming environments. 
% Live programming, in its various incarnations \cite{DBLP:journals/vlc/Tanimoto90,DBLP:conf/icse/Tanimoto13},
%has been and continues to be an active area of research and development.
% has been, and continues to be, an active area of research and development.

% \matt{ ``The problem that specifically motivates this paper'' is a
% really long noun phrase; the sentence is more interesting, IMO, when
% the subject is 'programming languages' and not that noun phrase.  }
%
The problem at the heart of this work is that
programming languages typically assign meaning only to {complete programs}, i.e. programs that are syntactically well-formed and free of static type and binding errors. A program editor, however, frequently encounters incomplete, and therefore meaningless, editor states. As a result, live feedback either ``flickers out'', creating a temporal gap, or it ``goes stale'', i.e. it relies on the most recent complete editor state, creating a perceptive gap because the feedback may not accurately reflect what the programmer is seeing in the editor. In some cases, these gaps can extend over substantial lengths of time.

% In some cases these gaps are momentary, like while the programmer
% %\IS entering
% is entering  
% a short expression. In other cases, these gaps can persist over substantial lengths of time, such as when there are many branches of a case analysis whose bodies are initially left blank or when the programmer makes a mistake.
% %
% Novice programmers, of course, make more mistakes \cite{mccauley2008debugging,fitzgerald2008debugging}.
% %
% %\matt{we are not really modeling a language with user type definitions -- what does the following sentence really add here? also, I just added this scenerio to the list above}
% The problem is particularly pronounced for languages with rich static type systems where certain program changes, such as a change to a type definition, can cause type errors to propagate throughout the program. Throughout the process of addressing these errors, the program text remains formally meaningless. 
% Overall, about 40\% of edits performed by Java programmers using Eclipse left the program text malformed \cite{popl-paper,6883030} and some additional number, which could not be determined from the data, were well-formed but ill-typed.

 % From the perspective of the language definition, these edit states are wholly meaningless, so editor services cannot rely on the same operations and reasoning principles that would be available to, for example, the compiler. 

% Despite these advances, there \IS a fundamental limitation which hinders
% the full realization of live programming: traditional programming languages
% cannot evaluate \emph{incomplete programs}, \ie{}, programs that are either
% syntactically ill-formed, or ill-typed.
% %
% Without a well-defined semantics to dictate the treatment of incomplete
% programs, various editor services may be provided in ad-hoc, restricted
% forms---if available at all---during the periods of time between complete
% program states.
% %
% %% ``flickering in and out''
% %
% %% In other words, the ``Edit-Compile-Run'' cycle routinely fails (either in the
% %% first or second steps) during the program editing process.
% %
% The resulting editor experience includes services ``flickering in and out,''
% thus introducing ``temporal and perceptive gaps'' because programmers often
% leave a program malformed or ill-typed for extended periods of time, \eg{}~as
% they think about what to enter at the cursor, or as they work on a different
% part of the program.
% %
% These times are, arguably, when rich, live editor services would be most helpful
% to the user.

%\matt{ Below, notice the broadening from ``the problem that motivates
%this paper is that incomplete lack dynamic meaing'' to just the
%general quesiton of ``incomplete program meaning''--- another reason
%to do my edit above, and remove that specific phrase.
%}
In recognition of this ``{gap problem}'', \citet{popl-paper} develop a static semantics (i.e. a type system) for incomplete 
functional programs, modeling them formally as typed expressions with \emph{holes}. 
Empty holes stand for missing expressions or types,
and non-empty holes operate as ``membranes'' around static type inconsistencies 
(i.e. they internalize the ``red underline'' that editors commonly display under a type inconsistency).
% For editor states into this language of incomplete programs, 
% editor services can reason about types and binding in many more situations than previously possible.
There are several ways to determine an expression with holes from the state of an editor. When the editor state is a text buffer, error recovery mechanisms can insert holes implicitly \cite{DBLP:journals/siamcomp/AhoP72,charles1991practical,DBLP:conf/oopsla/KatsJNV09}. 
Alternatively, the language might provide explicit notation for holes, so that the programmer can insert them either manually  
or semi-automatically via code completion \cite{Amorim2016}. For example, GHC Haskell supports the notation \li{_u} for empty holes, where \li{u} is an optional hole name \cite{GHCHoles}. Structure editors insert explicit holes automatically \cite{popl-paper}.

%
For the purposes of live programming, however, a static semantics does not suffice---%
we also need a corresponding dynamic semantics that specifies how to evaluate expressions with holes. This has been the focus of our recent research. %This paper addresses this need by developing
% a {dynamic semantics} for incomplete functional programs, 
% starting from the static semantics developed by \citet{popl-paper}.
%
%Our goal in this paper \IS to develop

The simplest approach would be to define a dynamic semantics that aborts with an error when evaluation reaches a hole. 
%%%%%%%%%%%% New par:
%
This mirrors a workaround that programmers commonly deploy: 
raising a placeholder exception, e.g. \lismall{raise Unimplemented}. 
GHC Haskell supports this mode of evaluation for programs with holes using the \lismall{-fdefer-typed-holes} flag.\footnote{
Without this flag, holes cause compilation to fail with an error message that reports information about each hole's type and typing context. 
Proof assistants like Agda \cite{norell:thesis,norell2009dependently} and Idris \cite{brady2013idris} also respond to holes in this way.
} 
Although better than nothing, this ``exceptional approach'' to evaluation with holes has limitations 
in the setting of a live programming environment because 
(1)~it provides no information about the behavior of the remainder of program, 
parts of which may not depend on the missing or erroneous expression (e.g. later cells in a lab notebook, or tests involving complete components of the program); 
(2)~it provides limited information about the dynamic state of the program where the exception occurs
(typically only a stack trace);  
and 
(3)~it provides no means by which to resume evaluation after filling a hole.

% Furthermore, exceptions can appear only in expressions, but we might also like to be able to evaluate programs that have type holes. Again, existing approaches do not support this situation well---GHC supports type holes, but compilation fails if type inference cannot automatically fill them \cite{GHCHoles}. 
% The static semantics developed by \citet{popl-paper} derives the machinery for 
% reasoning statically about type holes from gradual type theory, 
% identifying the type hole with the unknown type \cite{DBLP:conf/snapl/SiekVCB15,Siek06a}.
% %
% As such, we might look to the dynamic semantics from
% gradual type theory, 
% which selectively inserts dynamic casts as necessitated by missing type information to maintain type safety. 
% %However, this \IS again somewhat dissatisfying from the perspective of live programming because when a cast fails, evaluation again stops with 
% However, when a cast fails, evaluation again stops with 
% an exception and so the traditional gradual typing approaches still leave the live programming environment unable to provide rich, continuous feedback for the three reasons just enumerated.

% \parahead{Contributions}

This talk proposal is based on our ongoing effort to develop a dynamic semantics for incomplete functional programs, starting from the static semantics developed by \citet{popl-paper},  that addresses these three limitations of the exceptional approach. 
In particular, rather than stopping when evaluation encounters an expression hole, evaluation continues ``around'' the hole so that dynamic feedback about other parts of the program remains available. 
% For programs with unfillable type holes, casts are inserted as in gradual type theory \cite{DBLP:conf/snapl/SiekVCB15} and evaluation proceeds around failed casts in the same way. 
Uniquely, the system tracks the closure (i.e. variable environment) around each expression hole instance as evaluation proceeds. Hole closure tracking serves several purposes. 

First, when the programmer performs an edit that fills a hole, evaluation can resume from the paused, i.e. \emph{indeterminate}, evaluation state, rather than restarting evaluation from the beginning. We call this operation, which we describe briefly in Sec.~\ref{sec:examples}, \emph{fill-and-resume}. 

Second, the live programming environment can feed relevant information from the {hole closures} to the programmer, via various editor services, as they work to fill the holes in the program. 
This talk proposal will focus on demonstrating, by example, two such live, closure-driven editor services. We are integrating these two editor services into \Hazel, a live functional programming environment being developed by \citet{HazelnutSNAPL} (see Sec.~\ref{sec:examples} for more background on \Hazel). 

The first editor service, detailed in Sec.~\ref{sec:examples}, is \Hazel's \emph{live context inspector}, which displays static information together with relevant hole closure information. 
Programmers can rapidly switch between different closures for the selected hole and interactively explore nested hole closures, which arise from incomplete recursive functions. We give as an example two versions of quicksort.

The second editor service, which we propose by providing detailed mockups in Sec.~\ref{sec:palettes}, is \Hazel's \emph{live palettes}: user-defined direct manipulation user interfaces, which might themselves contain holes, that appear within
\Hazel programs and generate hole fillings. 
A live palette can evaluate an arbitrary expressions under the environment captured by a user-selected closure associated with the hole that the palette is tasked with filling in order to provide concrete feedback
about the dynamic implications of the choices the programmer has made in the palette UI. 

After describing these two editor services, we conclude in Sec.~\ref{sec:discussion} with a discussion of the state of our formal development (the details of which we omit from this submission) and future work.