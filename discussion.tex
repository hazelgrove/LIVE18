% !TEX root = hazel-LIVE2018.tex
\section{Discussion}
\label{sec:discussion}

Although we will not give the details here, we have developed a formal semantics,
mechanized in the Agda proof assistant, that
specifies how to evaluate programs with holes, and that assigns formal meaning to the fill-and-resume operation.

The editor component of this implementation defines a language of structured edit actions, 
based on the \Hazelnut structure editor calculus developed by \citet{popl-paper}, that inserts holes automatically to guarantee that
every editor state has some, possibly incomplete, type. 
The type safety invariant that we establish then guarantees that every editor state has dynamic meaning. Taken together, the result is a full solution to the gap problem, i.e. a proof-of-concept
live functional programming environment that provides rich static and dynamic feedback that never suffers from temporal or perceptive gaps.


Taken together, the result is a full solution to the gap problem: all of \Hazel's
editor services operate free of temporal or perceptive gaps, because every editor state
has non-trivial static and dynamic meaning. The editor services that make use of the 
tracked hole closures, in turn, provide \Hazel programmers with a uniquely 
\emph{live and direct} typed functional programming experience. 
